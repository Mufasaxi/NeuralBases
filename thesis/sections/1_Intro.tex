\newpage
\section{Introduction}
\label{sec:Introduction}
For years, the main driver for better chess programs was the development of more efficient algorithms such as MinMax, Alpha-Beta Pruning and Monte-Carlo Tree Search amongst many others \cite{Klein}. 

The sole purpose of each of them was to cut down on the time and resources used in order to traverse the massive move trees present during most positions of a chess game. The reason for that was due to the large branching factor of the game of chess, a position can have many millions of possible moves only at the third move of a game. Furthermore, the number of possible chess games is estimated to be around $10$\textsuperscript{120} \cite{Shan50}. With just these two facts alone, one could quickly notice that a brute force approach of traversing and evaluating each and every possible move in a given position would be futile and rather counterproductive, and that a better approach would focus on only a subset of moves (so called Candidate Moves) that could lead to a position unfolding in one's favour.

To cut down on the computing time needed to come up with the best move in a given position, resources such as Opening Books, Endgame Tablebases and several heuristics could be used in order to substantially decrease the size of the search tree. 

In this paper the main focus will be on the Endgame Tablebases. A game of chess is considered to be in the Endgame phase when there's only a few handful of pieces left on the board. The fewer the pieces that are left, the more plausible it would be to calculate each possible move in a given position and evaluate it. This is exactly what a Tablebase is, it is a large compilation of positions containing up to a certain amount of pieces on the board, and each position is then assigned a value that helps determine the evaluation of the position. At first glance, this may seem useful, as it would drastically cut down on the computations needed to evaluate a position algorithmic-ally, but it comes with a drawback just as significant in needing large amounts of storage space, and outside knowledge.

Alongside making the tablebase evaluations more accessible, using neural networks would help in answering the age old question of whether chess is a solved game or not. Solved in this context would mean, whether there's a known outcome that could be consistently reached given optimal play from both sides. In the case of positions with up to 7 pieces this is already determined. Neural networks could take a step beyond the available tablebases and incrementally increase the number of pieces while trying to extract a strategy to solve the types of positions that arise. Whether this approach could be extrapolated to at some point involve all pieces in the start position, this would definitely answer the question regarding the determinism of chess. This is of course far fetched with the current state of the art, but the idea could bare fruit as advancements in the field of AI take place over the coming years.

Due to that, the aim of this paper is to explore the use of Neural Networks as an alternative to tablebases. Both approaches will be discussed in detail, and their significant benchmarks would be compared. The different benefits and drawbacks would be considered and conclusions as to whether this would be a sensible approach to dealing with the issues of tablebases would be outlined at the end of this paper.
