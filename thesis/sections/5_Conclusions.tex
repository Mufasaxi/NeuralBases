\newpage
\section{Conclusions}
\label{sec:Conclusions}
This study has demonstrated the significant potential of neural networks as a viable alternative to traditional chess endgame tablebases. By employing a Convolutional Neural Network architecture specifically tailored for chess endgame positions, it was shown that it is possible to achieve high-quality predictions while drastically reducing storage requirements.
While the neural network approach does not yet match the perfect accuracy of tablebases, the trade-off between a small margin of error and a substantial reduction in storage space is compelling. This balance is particularly significant given the exponential growth in storage requirements for traditional tablebases as the number of pieces increases.
The results, though not flawless, serve as a proof of concept that opens up new avenues for research in this area. They invite further exploration into optimal neural network architectures that could potentially replace tablebases entirely. Future work in this direction could involve:
\begin{itemize}
\item Developing zero-knowledge models similar to AlphaZero, which learn purely from self-play without human knowledge
\item Exploring more advanced and deeper neural network architectures
\item Investigating the application of reinforcement learning techniques to improve model performance
\item Addressing the challenges of computational power and training time requirements
\end{itemize}
The rapid advancements in machine learning suggest that a complete replacement of tablebases by neural networks may be achievable in the foreseeable future.
Moreover, this approach could potentially extend beyond the current limitations of tablebases. While tablebases are currently limited to 7-piece endgames due to storage constraints, neural networks could theoretically handle more complex positions with additional pieces, opening up new possibilities in chess endgame analysis.
In conclusion, while there is still work to be done, this research highlights the potential for neural networks to revolutionize chess endgame analysis. As we continue to push the boundaries of machine learning and artificial intelligence, we may soon see a paradigm shift in how chess engines handle endgame positions, combining the efficiency of neural networks with the precision required for high-level chess play.